\logosection{\faGraduationCap}{教育经历}
\datedline{\textbf{东北大学}}{\dateRange{2023.09}{至今}}
计算机科学与工程学院 \quad 计算机科学与技术 \hfill 本科在读

\logosection{\faWrench}{项目经历}

\datedline{\textbf{TaskWizard (Open-AutoGLM Android Client)}}{\dateRange{2025.12}{至今}}
\datedline{\biInfo{基于 Android 原生的 AI Agent 自动化执行环境}{Kotlin, Jetpack Compose, Shizuku, AIDL}}{独立开发}
Github 链接:\href{https://github.com/tom-cat-mao/TaskWizard-Autoglm}{TaskWizard-Autoglm}
\begin{itemize}
  \item 重构 Open-AutoGLM 框架,将 PC 端 Python 逻辑完整移植为 Android 原生应用,实现无需 ADB/PC 的端侧自动化 (On-Device Agent)。
  \item 基于 Shizuku框架突破 Android 沙箱限制,通过 AIDL实现特权服务 IPC 通信,获取截屏、模拟点击与 Shell 执行等系统级权限。
  \item 开发自定义 IME (Input Method Service),解决跨应用文本注入难题;构建全局悬浮窗系统,实现可视化推理过程与人机接管 (Human-in-the-loop)。
  \item 设计严谨的上下文管理策略 (Context Management),在多轮对话中动态剥离视觉数据以优化 Token 消耗;实现基于协程的优雅任务调度与状态机管理。
  \item 集成 Retrofit 对接多模态大模型 API,结合系统元数据(当前应用包名、时间)与视觉截图构建多维 Prompt,提升模型决策准确率。
\end{itemize}

\datedline{\textbf{LLM-subtitle-translate 智能视频字幕系统}}{\dateRange{2026.01}{至今}}
\datedline{\biInfo{基于 DAG 任务编排的全栈视频处理平台}{Go, React, OpenAI API, FFmpeg}}{独立开发}
Github 链接:\href{https://github.com/tom-cat-mao/LLM-subtitle-Generate}{LLM-subtitle-Generate}
\begin{itemize}
  \item 设计并实现基于 DAG (有向无环图)的异步任务编排系统,能够自动化管理转录、翻译、对齐等长耗时 AI 任务依赖。
  \item 深度集成 OpenAI Whisper与 AssemblyAI通过 Go 协程并发处理多媒体流,封装 FFmpeg实现音频提取与波形数据生成。
  \item 前端采用 React + TypeScript构建专业级编辑器,利用 `react-virtuoso` 实现虚拟列表渲染,流畅编辑万行级字幕数据。
  \item 实现交互式可视化波形时间轴,支持毫秒级拖拽调整;支持 SRT/VTT/ASS 多格式互转与导出。
\end{itemize}

\datedline{\textbf{Greenlight 电影评价系统 API}}{\dateRange{2025.04}{2025.07}}
\datedline{\biInfo{高并发生产级 RESTful API 后端服务}{Go, PostgreSQL, Redis, Docker}}{独立开发}
Github 链接: \href{https://github.com/tom-cat-mao/Golang-Greenlight-API}{greenlight}
\begin{itemize}
  \item 设计符合 JSON API 规范的 RESTful接口,实现基于 Token Bucket算法的分布式限流 (Rate Limiting) 中间件。
  \item 引入 Redis缓存层与数据库乐观锁机制 (Optimistic Concurrency Control),有效解决高并发场景下的数据一致性问题。
  \item 构建清晰的分层架构 (Layered Architecture),实现基于 **JWT/PASETO** 的无状态认证体系,集成 SMTP 服务实现异步业务流程。
  \item 采用 `log/slog` 实现结构化日志追踪,结合 `expvar` 暴露运行时应用指标 (Metrics);配置 Graceful Shutdown 保障服务高可用。
\end{itemize}

\logosection{\faCogs}{专业技能}

\begin{itemize}[parsep=0.5ex]
  \item \textbf{编程语言}: Kotlin, Go (熟练), Java, Python, TypeScript/JavaScript, SQL, C++
  \item \textbf{Android开发}: Jetpack Compose, AIDL, Shizuku, Retrofit, Room, MVVM, Coroutines
  \item \textbf{AI/Agent}: Prompt Engineering, Agent Workflow (ReAct/DAG), OpenAI API, 多模态交互设计
  \item \textbf{后端/全栈}: Gin/Echo, Redis, PostgreSQL, Docker, React, Linux (Arch)
  \item \textbf{工具链}: Git, Make, Vim/Neovim, GitHub Actions
\end{itemize}

\logosection{\faInfo}{其他}

\begin{itemize}[parsep=0.5ex]
  \item \textbf{语言}: 英语 (IELTS 7.0) - 具备流畅的英文文档阅读与技术交流能力
  \item \textbf{GitHub}: \href{https://github.com/tom-cat-mao}{github.com/tom-cat-mao}
\end{itemize}

%%%% 如果多页简历,可以手动在适当位置插入 \newpage 或者 \clearpage 开始新一页
